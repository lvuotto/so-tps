\documentclass[a4paper]{article}
\usepackage[spanish]{babel}
\usepackage[utf8]{inputenc}
\usepackage{fancyhdr}
\usepackage{charter}   % tipografia
\usepackage{graphicx}
\usepackage{makeidx}

\usepackage{float}
\usepackage{amsmath, amsthm, amssymb}
\usepackage{amsfonts}
\usepackage{sectsty}
\usepackage{wrapfig}
\usepackage{listings}
\usepackage{caption}

\usepackage{hyperref} %las entradas del índice tienen links
\hypersetup{
    colorlinks=true,
    linktoc=all,
    citecolor=black,
    filecolor=black,
    linkcolor=black,
    urlcolor=black
}

\input{codesnippet}
\input{page.layout}
\usepackage{underscore}
\usepackage{caratula}
\usepackage{url}

\usepackage{color}
\usepackage{clrscode3e} % para el pseudocodigo




\begin{document}

\lstset{
  language=C++,
  backgroundcolor=\color{white},   % choose the background color
  basicstyle=\footnotesize,        % size of fonts used for the code
  breaklines=true,                 % automatic line breaking only at whitespace
  captionpos=b,                    % sets the caption-position to bottom
  commentstyle=\color{mygreen},    % comment style
  escapeinside={\%*}{*)},          % if you want to add LaTeX within your code
  keywordstyle=\color{blue},       % keyword style
  stringstyle=\color{mymauve},     % string literal style
}

\thispagestyle{empty}
\materia{Sistemas Operativos}
\submateria{Segundo Cuatrimestre de 2014}
\titulo{Trabajo Práctico II}
%\subtitulo{}
\integrante{Caravario, Martín}{470/12}{martin.caravario@gmail.com}
\integrante{Hosen, Federico}{825/12}{fhosen@hotmail.com}
\integrante{Vuotto, Lucas}{385/12}{lvuotto@dc.uba.ar}

\maketitle
\newpage

\thispagestyle{empty}
\vfill
\thispagestyle{empty}
\vspace{1.5cm}
\tableofcontents
\newpage


%\normalsize
\newpage

\section{Introducción}

El objetivo de este trabajo práctico es implementar un nuevo diseño de una
simulación de simulacro de evacuación.

Se desea pasar de un modelo  \textit{mono-thread} a uno
\textit{multi-thread}, para poder simular que varios alumnos se muevan por
el aula simultaneamente.

Para esta nueva implementación se pide utilizar la biblioteca
\textit{Pthreads}, restringiéndonos únicamente a los mutexes y variables de
condición provistas por ésta.

En este informe detallaremos la implmentación realizada, y justificaremos
las decisiones que fuimos tomando a la hora de resolver el trabajo práctico.

Dicha explicación y justificación estarán en la sección que se encuentra a
continuación, en forma de un único texto.



\subsection{Detalles de implementación}
\newpage



\subsection{Paralelismo}

\subsection{Deadlock}

\end{document}
